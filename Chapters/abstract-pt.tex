%!TEX root = ../template.tex
%%%%%%%%%%%%%%%%%%%%%%%%%%%%%%%%%%%%%%%%%%%%%%%%%%%%%%%%%%%%%%%%%%%%
%% abstract-pt.tex
%% NOVA thesis document file
%%
%% Abstract in Portuguese
%%%%%%%%%%%%%%%%%%%%%%%%%%%%%%%%%%%%%%%%%%%%%%%%%%%%%%%%%%%%%%%%%%%%

\typeout{NT FILE abstract-pt.tex}%

\paragraph{}Ao longo dos anos, o mundo da tecnologia tem-se expandido em todos os possíveis campos, 
e a agricultura não é exceção. O uso de tecnologia na Agricultura Inteligente tem vindo a aumentar, 
com redes de Internet of Things, sistemas de Inteligência Artificial para gestão e previsão, e robótica para 
automatizar processos, realizando as tarefas repetitivas que os humanos preferem não fazer, como 
aplicar pesticidas, colher produtos e avaliar o estado dos campos, além de algumas que os humanos 
não conseguem fazer, como a vigilância de grandes áreas agrícolas. Estes desenvolvimentos não só 
facilitam as tarefas, como também combatem problemas contínuos no ambiente agrícola, como a falta de 
jovens dispostos a realizar trabalhos duros no campo e o aumento da população mundial, que irá 
intensificar a necessidade de produção de alimentos.

\paragraph{}Este trabalho centra-se no desenvolvimento de um trator com reboque, capaz de navegar 
autonomamente por um campo enquanto aplica pesticidas diretamente sobre os produtos. 
O foco é na navegação do robô, investigando o uso da robótica na Agricultura de Precisão, 
explorando métodos de planeamento de trajetórias para robôs, como Voronoi Graphs, Visibility Graphs, 
abordagens baseadas em amostragem como o Rapidly exploring Random Trees e o Probabilistic Roadmap, algoritmos bio heurísticos como o Ant Colony Optimization, 
Particle Swarm Optimization e Genetic Algorithm, e mencionando ainda métodos baseados em aprendizagem. Este trabalho também irá rever alguns 
métodos de controlo incluídos na biblioteca nav2 do ROS2, como o Dynamic Windows Approach, Pure Pursuit e o Model Predictive Controller. Esta 
tarefa já é desafiante por si só, no entanto, com a adição do reboque ao sistema, a tarefa torna-se 
ainda mais complexa, pois será necessário considerar as dinâmicas do reboque. No final, os métodos 
escolhidos para serem testados serão o Voronoi Graphs com o algoritmo A* para o planeamento de 
trajetórias e o controlador Pure Pursuit para o seguimento da trajetória.

\paragraph{}O sistema trator-reboque foi escolhido, apesar da sua dificuldade e escassez de documentação, 
devido à sua modularidade, permitindo que o seu funcionamento seja, por regra, independente e reutilizável.


\keywords{ Agricultura Inteligente \and 
trator-reboque \and
IoT \and 
Inteligência Artificial \and 
Robótica \and 
Planeamento de trajetórias \and 
Controlo de robôs \and 
ROS2 \and 
Voronoi Graphs \and 
A* \and 
Pure Pursuit }
% to add an extra black line

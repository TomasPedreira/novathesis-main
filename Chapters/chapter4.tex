%!TEX root = ../template.tex
%%%%%%%%%%%%%%%%%%%%%%%%%%%%%%%%%%%%%%%%%%%%%%%%%%%%%%%%%%%%%%%%%%%%
%% chapter4.tex
%% NOVA thesis document file
%%
%% Chapter with lots of dummy text
%%%%%%%%%%%%%%%%%%%%%%%%%%%%%%%%%%%%%%%%%%%%%%%%%%%%%%%%%%%%%%%%%%%%

\typeout{NT FILE chapter4.tex}%

\chapter{Conclusion}
\label{cha:conclusion}

\paragraph{}This dissertation plan focuses on the motion planning and control of a 
tractor-trailer system for smart agriculture. The document begins by addressing 
the growing challenges in agriculture, such as the need for increased food 
production, labor shortages, and environmental concerns. These challenges motivate 
the integration of robotics into agriculture, particularly autonomous systems 
designed to optimize efficiency, reduce human intervention, and improve 
sustainability.


The literature review provides a detailed analysis of the current state of mobile 
robotics in agriculture, motion planning methods, and robot control techniques. 
Various approaches, such as cell decomposition, sampling-based algorithms, 
bio-inspired techniques, and learning-based methods, are explored in the context of 
path planning. The review also examines control methods like Pure Pursuit, Model 
Predictive Control, and the Dynamic Windows Approach. Furthermore, it highlights 
related works on tractor-trailer systems and discusses their applications, 
challenges, and potential for improvement. This comprehensive review establishes a 
foundation for the methodology proposed in this work.

The planning chapter outlines the steps required to develop the dissertation. 
These include the development of a simulation environment, the design 
and tuning of planning and control algorithms, and their implementation and testing 
in real-world conditions. The planning process also considers the practical 
challenges of transitioning from simulation to physical systems, such as sensor 
errors and environmental variability. A detailed schedule ensures timely progress 
through each stage of the work.

The results publishing plan identifies key conferences and journals where the 
outcomes of this work can be disseminated. Conferences like IEEE ICRA, ICMA, and 
IROS were chosen for their relevance to robotics and intelligent systems, while 
journals such as Expert Systems with Applications, Applied Sciences, and the 
Journal of Intelligent and Robotic Systems provide suitable platforms for 
presenting the research findings to both academic and industrial audiences.

In summary, this dissertation addresses the pressing need for autonomous systems 
in agriculture by proposing a tractor-trailer robot capable of autonomous 
navigation and pesticide application. The proposed system combines modularity, 
efficiency, and precision, making it a viable solution for current challenges in 
agriculture. The literature review and planning chapters provide the theoretical 
and practical foundation for the project, while the results dissemination plan 
ensures the research will contribute meaningfully to the field of robotics and 
smart agriculture.

Looking ahead, the successful implementation of this system will demonstrate 
the potential of robotics in transforming agricultural practices. By reducing 
manual labor, optimizing resource use, and improving the accuracy of agricultural 
processes, this work aims to make a significant impact on the sustainability and 
productivity of modern farming.


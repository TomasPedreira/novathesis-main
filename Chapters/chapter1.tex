%!TEX root = ../template.tex
%%%%%%%%%%%%%%%%%%%%%%%%%%%%%%%%%%%%%%%%%%%%%%%%%%%%%%%%%%%%%%%%%%%
%% chapter1.tex
%% NOVA thesis document file
%%
%% Chapter with introduction
%%%%%%%%%%%%%%%%%%%%%%%%%%%%%%%%%%%%%%%%%%%%%%%%%%%%%%%%%%%%%%%%%%%

\typeout{NT FILE chapter1.tex}%

\chapter{Introduction}
\label{cha:introduction}
\paragraph{}In this chapter, there will be a background explanation of the proposed problem, 
the motivation for a solution and the document structure.

\section{Background}
\label{sec:background} 
\paragraph{}Agriculture is one of the most essential industries, providing a source of food for the growing population 
around the globe. However, the industry faces several challenges, such as the need to increase food production, 
for example, according to the Food and Agriculture Organization (FAO) of the United Nations, around eight hundred million 
people are undernourished, and two thirds of them live in Asia. For example, in India, over 70\% of the population 
is dependant on agriculture for their livelihood ~\cite{agriIndia}, and, due to lack of the development of 
the countries’agricultural processes, most farmers use very traditional methods of farming, which are not efficient 
requiring a lot of time and manual labour. To overcome these issues and to meet the growing demand for food,
the agriculture industry is turning to technology, such as robotics, to improve efficiency and productivity.
\paragraph{}The use of technologies like AI, Internet of Things, and robotics facilitates the automation of 
several tasks, such as planting, weeding, harvesting, forecasting, and monitoring which takles the issues of 
labour shortages and allowing for easier expansion solving the need for more food production. One particularly important application of robotics in agriculture is in pesticide spraying. 
Traditionally, pesticide application has been a labour-intensive and slow task. Autonomous pesticide 
spraying robots address these issues by applying pesticides more accurately, reducing chemical waste, and 
minimizing human exposure to harmful substances.


\section{Motivation}
\label{sec:motivation}
\paragraph{}The motivation behind the development of a pesticide spraying robot is the need to improve agricultural 
processes to meet the demand for food production and safety. By integrating these robots in Smart Farming, 
farmers can optimize pesticide usage, reducing the amount of pesticide wasted and overexposure to 
chemicals. The solution proposed in this work is to develop a trailer-tractor system, where the tractor will be responsible for 
towing a trailer with the pesticide spraying system. The advantage of this system is its modularity with each part acting 
independently, allowing for easier reuse and maintenance. This solution will also tackle the issues of labour 
shortages diminishing the need for manual spraying. The main technical issue with these systems is their non-holomic 
charecteristics, which makes path planning and control more difficult.


\section{Document Structure}
\label{sec:documentstruct}
\paragraph{}This document is divided into 4 chapters. The first chapter is the introduction, 
where the problem is presented along with the motivation for the work and the document structure. 
The second is the state of the art review, where a contextualization of mobile robotics in agriculture is made 
along with the most popular methods for path planning and robot control, and some previous works 
on the trailer-tractor system. The third chapter is the planning and schedule of the work, 
where the work proposal, schedule and results publishing plan are presented. The last chapter is the 
conclusion, where a summary of the document is presented.